%%% vim:ai:tw=72:
\chapter{Intel CET Extension}

Intel CET (Control-flow Enforcement Technology) Extension includes:

\begin{description}
  \item[IBT (Indirect Branch Tracking)] Branch protection to defend
    against Jump/Call Oriented Programming.
  \item [SHSTK (Shadow Stack)] Return address protection to defend
    against Return Oriented Programming,
\end{description}

\section{Program Loading}

\subsection{Process \code{GNU_PROPERTY_X86_FEATURE_1_IBT}}
\label{ibt}

On an IBT capable processor, the following steps should be taken:

\begin{enumerate}
  \item
    \begin{sloppypar}
      When loading an executable without interpreter, enable IBT if
      \code{GNU_PROPERTY_X86_FEATURE_1_IBT} is set on the executable.
    \end{sloppypar}
  \item
    \begin{sloppypar}
      When loading an executable with an interpreter, enable IBT if
      \code{GNU_PROPERTY_X86_FEATURE_1_IBT} is set on the interpreter.
      The interpreter should disable IBT if
      \code{GNU_PROPERTY_X86_FEATURE_1_IBT} isn't set on the executable
      or any shared objects loaded via the \code{DT_NEEDED} tag.
    \end{sloppypar}
  \item
    \begin{sloppypar}
      After IBT is enabled, when loading a shared object
      \footnote{The legacy code page bitmap isn't used since it can't
      cover the jitted code generated by legecy JIT engines.}
      without \code{GNU_PROPERTY_X86_FEATURE_1_IBT}:
    \end{sloppypar}
    \begin{enumerate}
      \item If legacy interwork is allowed, IBT should be disabled.
      \item If legacy interwork isn't allowed, it causes an error.
    \end{enumerate}
\end{enumerate}

\subsection{Process \code{GNU_PROPERTY_X86_FEATURE_1_SHSTK}}
\label{shstk}

On a SHSTK capable processor, the following steps should be taken:

\begin{enumerate}
  \item
    \begin{sloppypar}
      When loading an executable without interpreter, enable SHSTK if
      \code{GNU_PROPERTY_X86_FEATURE_1_SHSTK} is set on the executable.
    \end{sloppypar}
  \item
    \begin{sloppypar}
      When loading an executable with an interpreter, enable SHSTK if
      \code{GNU_PROPERTY_X86_FEATURE_1_SHSTK} is set on the interpreter.
      The interpreter should disable SHSTK if
      \code{GNU_PROPERTY_X86_FEATURE_1_SHSTK} isn't set on the executable
      or any shared objects loaded via the \code{DT_NEEDED} tag.
    \end{sloppypar}
  \item
    \begin{sloppypar}
      After SHSTK is enabled, when loading a shared object without
      \code{GNU_PROPERTY_X86_FEATURE_1_SHSTK}:
    \end{sloppypar}
    \begin{enumerate}
      \item If legacy interwork is allowed, SHSTK should be disabled.
      \item If legacy interwork isn't allowed, it causes an error.
    \end{enumerate}
\end{enumerate}

\section{Dynamic Linking}

To support IBT, linker should generate a IBT-enabled PLT (Procedure
Linkage Table) (see figure \ref{ibt_small_med_plt}) together with a second
PLT (see figure \ref{ibt_sec_small_med_plt}).

\begin{figure}[H]
\Hrule
\caption{The IBT-enabled PLT (small and medium models)}
\label{ibt_small_med_plt}
\begin{footnotesize}
\begin{verbatim}
  .PLT0: pushq    GOT+8(%rip)             # GOT[1]
         bnd jmp  *GOT+16(%rip)           # GOT[2]
         nopl     (%rax)
  .PLT1: endbr64                          # 16 bytes from .PLT0
         pushq    $index1
         bnd jmp  PLT0
         nop
  .PLT2: endbr64                          # 16 bytes from .PLT1
         pushq    $index2
         bnd jmp  .PLT0
         nop
  .PLT3: ...
\end{verbatim}%$
\end{footnotesize}
\Hrule
\end{figure}

\begin{figure}[H]
\Hrule
\caption{The Second IBT-enabled PLT (small and medium models)}
\label{ibt_sec_small_med_plt}
\begin{footnotesize}
\begin{verbatim}
  .SPLT1: endbr64
          bnd jmp  *name1@GOTPCREL(%rip)
          nopl     0x0(%rax,%rax,1)
  .SPLT2: endbr64
          bnd jmp  *name2@GOTPCREL(%rip)  # 16 bytes from .SPLT1
          nopl     0x0(%rax,%rax,1)
  .SPLT3: ...                             # 16 bytes from .SPLT2
\end{verbatim}%$
\end{footnotesize}
\Hrule
\end{figure}

The BND prefix is added so that the IBT-enabled procedure linkage table is
also compatible with MPX.

When the IBT-enabled procedure linkage table is used, the initial value of
the global offset table entry for the external function is the address of
the corresponding entry of the second procedure linkage table.

\subsection{IBT-enabled PLTs For ILP32}

Since MPX isn't supported in ILP32 programming model, there is no
need to add the BND prefix.

\begin{figure}[H]
\Hrule
\caption{The IBT-enabled PLT For ILP32}
\label{ibt_small_med_plt_ipl32}
\begin{footnotesize}
\begin{verbatim}
  .PLT0: pushq    GOT+8(%rip)             # GOT[1]
         jmp      *GOT+16(%rip)           # GOT[2]
         nopl     0x0(%rax)
  .PLT1: endbr64                          # 16 bytes from .PLT0
         pushq    $index1
         jmp  PLT0
         xchg %ax,%ax
  .PLT2: endbr64                          # 16 bytes from .PLT1
         pushq    $index2
         jmp  .PLT0
         xchg %ax,%ax
  .PLT3: ...
\end{verbatim}%$
\end{footnotesize}
\Hrule
\end{figure}

\begin{figure}[H]
\Hrule
\caption{The Second IBT-enabled PLT For ILP32}
\label{ibt_sec_small_med_plt_ipl32}
\begin{footnotesize}
\begin{verbatim}
  .SPLT1: endbr64
          jmp  name1@GOTPCREL(%rip)
          nopw 0x0(%rax,%rax,1)
  .SPLT2: endbr64
          jmp  name2@GOTPCREL(%rip)       # 16 bytes from .SPLT1
          nopw 0x0(%rax,%rax,1)
  .SPLT3: ...                             # 16 bytes from .SPLT2
\end{verbatim}%$
\end{footnotesize}
\Hrule
\end{figure}

%%% Local Variables:
%%% mode: latex
%%% TeX-master: "abi"
%%% End:
